\section{Introduction}
Nowadays, in order to solve problems modelled by partial differential equations (PDE), numerical methods are widely applied. This numerical process takes two steps: discretisation and solution. Discretisation process turns a continuous PDE into a discrete algebraic equation, whose solution is an approximation of the solution for the continuous problem. After discretisation, the obtained linear equation system needs to be solved by a solver. These two process form a full cycle for one numerical method. Recently, a unique discretisation method: mimetic spectral element method (MSEM) is studied by us in depth, showing great potential. Until now, the discretisation technique is studied to perform in all kinds of problems in physics. However, in order to further extend the power of this method, the solution process also needs to be inspected. Usually, between discretisation and solution, an intermediate step called preconditioning, is performed. This step is to, naively speaking, tune the discrete system obtained from discretisation process in its optimal condition, such that the solver can perform the best precision and fastest convergence rate. \\

In this literature report, a brief review of mimetic finite/spectral element method will be presented in Section 2. In Section 3, a comprehensive review will be given, exploring some of the most important and relevant preconditioning techniques in history. Some important conclusions within author's view will be made. \\    





















  